%!TEX root = ../../report.tex
\subsection{JavaScript}

A framework \cite{feldthaus2011tool} for refactoring JavaScript programs was created because there are few refactoring tools for JavaScript. 
A problem that might be responsible for the additional difficulty that the refactoring tools have to deal with, when compared with refactoring tools made for static languages. 
E.g. when refactoring JavaScript the refactoring tools do not have information about the bindings in compile time.

%TODO add the refactoring that they create!


The framework uses pointer analysis to help define a set of general analysis queries. 
It also uses under-approximations and over-approximations of sets in a safe way and uses preconditions.
In order to be able to create a correct refactoring operation these conditions are expressed in terms query analyses, 
If it is not possible for to guarantee behavior preservations, the refactoring operation is prevented.
With this approach it is possible to be sure when a refactoring operation is valid but it has the catch of not making every possible refactoring operations because it is an approximation to the set.

%To prove the concept it was implemented three refactoring operations, namely the rename, encapsulate property and extract module.

%Talk about the tests made, that count what it counts.

Because it uses approximations it has a certain percentage of refactoring operations that the framework will unjustifiably reject.
While a manual programmer doing that refactoring would be able to do. 
However, after testing with 50 JavaScript programs, the overall unjustified rejections were of 6.2\%. 
The rejections due to imprecise preconditions represent 8.2\%.
Unjustified rejections due to imprecise pointer analysis were of 5.9\% for the rename and 7.0\% for the encapsulate property. 
