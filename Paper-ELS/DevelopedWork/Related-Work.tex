\section{Related-Work -Change} %Rewrite/Create a Story
A refactoring tool \cite{griswold1991program} for scheme, implemented in Lisp that
uses two forms of information, AST and PDG (program dependence Graph).
The AST represents the abstract syntactic structure of the program.
While the PDG explicitly represents the key relationship of dependence between
operations in the program. %control flow!
The graph vertices's represent program operations and the edges represent the flow of
data and control between operations.
However the PDG only has information of dependencies of the program. And if there
are two semantically unrelated statements they could be placed arbitrarily with
respect to each other.
%Whitout the constraints given by the AST two semantically unrelated statements
%could be placed arbitrarily with respect to each other. %rewrite!
Using the AST as the main representation of the program ensures that statements
are not arbitrarily reorder.
And the PDG is only used as a notation to prove that transformations preserve
the meaning and as quick way to retrieve needed dependence information.
Contours are used with the PDG providing scope information, non existent in the PDG,
to help reason about transformations in the PDG.
%Rewrite, talk about the scopes
With these structures it is possible to have a single formalism to reason effectively
about flow dependencies and scope structure.

HaRe \cite{thompson2005refactoring} is a refactoring tool for Haskell that
integrates with Emacs and Vim.
%(abstract syntax tree)
The HaRe system uses an AST of the program to be
refactored in order to reason about the transformations to do.
The system has also a token stream in order to preserve the comments and the
program layout by keeping information about the source code location and the comments of all tokens.
It retrieves scope information from the AST, that allows to have refactoring operations that
require binding information of variables. %so a def-use-relation?
The system also allows the users to design their own refactoring operations
using the HaRe API.
