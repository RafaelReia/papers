/section{Related Work}
%Griswold
Griswold \cite{griswold1991program} proved that meaning-preserving restructuring
can substantively reduce the maintenance cost of a system. %How??
A prototype was created to prove the concept, by creating restructuring operations,
 for the Scheme programming language, implemented in Common Lisp.
The prototype was developed for Scheme because of its imperative features, simple %ingles
syntax and was available an implementation in Common Lisp of a Scheme PDG (program dependence graph).
The prototype had simple restructuring operations to prove the concept, such as:
 moving an expression, renaming a variable, abstracting an expression, extracting a function, scope-wide function replacement, adding a reference indirection and adding looping to list references.

%%REWRITE THIS
The Refactoring tool uses two forms of information, Countours and PDG (program dependence Graph),
that are used to reason about the program.
%In order to be able to correctly implement these operations it was used contours and a PDG.

It uses Contours as the main program representation.
Contours are an abstraction of the essential semantic properties that the AST
represents in an efficient and complete form.
And it uses the PDG to ensure [Insert Stuff Here]

Whereas the PDG explicitly represents the key relationship of dependence between
the operations in the program. %control flow!
The PDG is used because simple graph algorithms can extract this information and
it has been a popular program representation for aiding program parallelization,
optimization and version merging.
These features combined with the right semantic support make the PDGs a good
foundation for preserving meaning during restructuring.
With these structures it is possible to combine them and have a single formalism
to reason effectively about flow dependencies and scope structure.



%Hare confirm and check this.

HaRe \cite{thompson2005refactoring} is a refactoring tool for Haskell that integrates with Emacs and Vim.
This tool was made for being used instead of being a proof of concept prototype
and it is implemented in Haskell.
The system also allows the users to design their own refactoring operations
using the HaRe API.

The HaRe system uses an AST (abstract syntax tree) of the program to be
refactored in order to reason about the transformations to do.
The system has also a token stream in order to preserve the comments and the
program layout by keeping information about the source code location and the comments of all tokens.
%Smalltalk Refactoring Browser

The Refactoring Browser \cite{roberts1997refactoring} is a refactoring tool for
Smalltalk programs which goal was to make refactoring known and used by everyone.

To do that they implemented the refactoring browser with the concern that the
 refactoring operations did by the programmer using the refactoring browser needed to be faster than by hand.

It was considered that the user of this tool is intelligent therefore automated
refactorings would not suit them.
Automated refactorings do not suit the user because they could generate code that
 would not make sense in the domain.
For example, the tool would generate new classes in order to eliminate duplicated
code, by creating an abstract class, which might not make sense in the domain.
Instead of doing that, the tool points out possible refactoring operations and
lets the user decide whether or not to do those operations.

In order to ensure behavior preservation the tool checks the preconditions
of each refactoring operation before execution.
However, there are some conditions that are more difficult to determine statically,
 such as dynamic typing and relationships cardinality between objects.
Instead of checking the precondition statically the refactoring browser checks
the preconditions dynamically.

The preconditions checks are done using method wrappers to collect runtime information.
The Refactoring Browser starts by doing the refactoring operation and then it
adds a wrapper method to the original method.
While the program is running, the wrapper detects the source code that called
the original method and changes it for the new method.
For example, in the rename operation, after applying the rename and while the
program is running, whenever the old method is called, the browser suspends the
execution and changes the code that called the old method, so that it now calls the new method.
The problem of this approach is that the dynamically analysis is only as good
as the test suit used by the programmer.

%Bicycle Repair Man

\subsection{Python}

The following section presents two refactoring tools for Python.
It starts with Bicycle-Repair-Man\footnote{https://pypi.python.org/pypi/bicyclerepair/0.7.1},
 a refactoring tool that attempts to create a refactoring browser.
Afterwards it presents Rope\footnote{https://github.com/python-rope/rope}, a
refactoring tool that works like a Python library.

\subsubsection{Bicycle Repair Man}

 is a Refactoring Tool for Python written in Python.
 This refactoring tool can be added to IDEs and editors, such as Emacs, Vi, Eclipse,
  and Sublime Text. However, this tool did not improve since 2004.

Bicycle Repair Man is an attempt to create the refactoring browser functionality for
 Python and has the following refactoring operations: extract method, extract variable, inline variable, move to module and rename.

The tool has an AST to represent the program and a database that has information
about the several program entities and dependencies information.
Bicycle Repair Man does its own parsing so it replaces the Python's parser with
its own wrapper to be easier to develop the refactoring operations.

%Rope

\subsubsection{Rope}

 is a Python refactoring tool written in Python, which works like a Python library.
In order to make it easier to create refactoring operations Rope assumes that a
Python program only has assignments and functions calls. %(can this be a bad thing?)
The tool can easily get information about the assignments.
However for functions calls it is necessary to have other approaches in order to
obtain the necessary information.

Rope uses a Static Object Analysis which analyses the modules or scope to get
information about functions.
Rope only analyses the scopes when they change and it only analyses the modules
when asked by the user, because this approach is time consuming.

The other approach is the Dynamic Object Analysis that starts working when a
module is running.
Then, Rope collects information about parameters passed to and returned from
functions in order to get all the information needed.
The main problem is that this approach is slow while collecting information,
but not while accessing the information.

Rope stores the information collected by the analysis in a database.
If Rope needs the information and there is nothing on the database the Static
object inference starts trying to infer the object information.

Rope uses an AST in order to store the syntax information about the programs.
