\section{Related Work}
\label{sec:Related-Work}

There are many refactoring tools available.  The large majority of
these tools were designed to deal with large statically-typed
programming languages such a Java or C++ and are integrated in the
complex IDEs typically used for the development of complex software
projects, such as Eclipse or Visual Studio.

On the other hand, it is a common practice to start teaching beginner
programmers using dynamically-typed programming languages, such as
Scheme, Python, or Ruby, using simple IDEs.  As a result, our focus
was on the dynamic programming languages which are used in
introductory courses and, particularly, those that promote a
functional programming paradigm.

In the next sections, we present an overview of the refactoring tools
that were developed for the languages used in introductory programming
courses.

\subsection{Scheme}

In its now classical work \cite{griswold1991program}, Griswold
presented a refactoring tool for Scheme that uses two different kinds
of information, namely, an Abstract Syntax Tree (AST) and a Program
Dependence Graph (PDG).

The AST represents the abstract syntactic structure of the program,
while the PDG explicitly represents the key relationship of dependence
between operations in the program. %control flow!
The graph vertices's represent program operations and the edges
represent the flow of data and control between operations.  However,
the PDG only has dependency information of the program and relying
only in this information to represent the program could create
problems.  For example, two semantically unrelated statements can be
placed arbitrarily with respect to each other.  Using the AST as the
main representation of the program ensures that statements are not
arbitrarily reordered, allowing the PDG to be used to prove that
transformations preserve the meaning and as a quick way to retrieve
needed dependence information.  Additionally, contours are used with
the PDG to provide scope information, which is non existent in the
PDG, and to help reason about transformations in the PDG.
%Rewrite, talk about the scopes
With these structures it is possible to have a single formalism to
reason effectively about flow dependencies and scope structure.

% \subsection{Haskell}

% Although Haskell is a statically-typed language, its use in
% introductory programming justifies a brief discussion.  HaRe
% \cite{thompson2005refactoring} is a refactoring tool for Haskell that
% integrates with Emacs and Vim.
% %(abstract syntax tree)
% The HaRe system uses an AST of the program to be refactored in order
% to reason about the transformations to do.  The system has also a
% token stream in order to preserve the comments and the program layout
% by keeping information about the source code location and the comments
% of all tokens.  It retrieves scope information from the AST, allowing
% it to have refactoring operations that require binding information of
% variables. %so a def-use-relation?
% The system also allows the users to design their own refactoring
% operations using the HaRe API.

\subsection{Python}

Rope\cite[p.~109]{govindaraj2015test} is a Python refactoring tool
written in Python, which works like a Python library.  In order to
make it easier to create refactoring operations, Rope assumes that a
Python program only has assignments and function
calls. %(can this be a bad thing?)
Thus, by limiting the complexity of the language it reduces the
complexity of the refactoring tool.

Rope uses a Static Object Analysis, which analyses the modules or
scopes to get information about functions.  Because its approach is
time consuming, Rope only analyses the scopes when they change and it
only analyses the modules when asked by the user.

Rope also uses a Dynamic Object Analysis that requires running the
program in order to work.  The Dynamic Object Analysis gathers type
information and parameters passed to and returned from functions.  It
stores the information collected by the analysis in a database.  If
Rope needs the information and there is nothing on the database the
Static object inference starts trying to infer it.  This approach
makes the program run much slower, thus it is only active when the
user allows it. %TODO improve "user decides"
Rope uses an AST in order to store the syntax information about the programs.\\

Bicycle Repair
Man\footnote{https://pypi.python.org/pypi/bicyclerepair/0.7.1} is
another refactoring tool for Python and is written in Python itself.  This
refactoring tool can be added to IDEs and editors, such as Emacs, Vi,
Eclipse, and Sublime Text.  It attempts to create the refactoring
browser functionality for Python and has the following refactoring
operations: extract method, extract variable, inline variable, move to
module, and rename.

The tool uses an AST to represent the program and a database to store
information about several program entities and their dependencies. \\


Pycharm Educational
Edition,\footnote{https://www.jetbrains.com/pycharm-edu/} or Pycharm
Edu, is an IDE for Python created by JetBrains, the creator of
IntelliJ.  The IDE was specially designed for educational purposes,
for programmers with little or no previous coding experience.  Pycharm
Edu is a simpler version of Pycharm community which is the free python
IDE created by JetBrains.  It is very similar to their complete IDEs
and it has interesting features such as code completion and
integration with version control tools.  However, it has a simpler
interface than Pycharm Community and other IDEs such as Eclipse or
Visual Studio. %good for beginners, explain

Pycharm Edu integrates a python tutorial and supports teachers that
want to create tasks/tutorials for the students.  However, the
refactoring tool did not received the same care as the IDE itself.
The refactoring operations are exactly the same as the Pycharm
Community IDE which were made for more advanced users.  Therefore, it
does not provide specific refactoring operations to beginners.  The
embedded refactoring tool uses the AST and the dependencies between
the definition and the use of variables, known as def-use relations.

\subsection{Javascript}

There are few refactoring tools for JavaScript but there is a
framework \cite{feldthaus2011tool} for refactoring JavaScript
programs. %FIXME I do not like it
In order to guarantee the correctness of the refactoring operation,
the framework uses preconditions, expressed as query analyses provided
by pointer analysis. %TODO explain better!
Queries to the pointer analysis produces over-approximations of sets
in a safe way to have correct refactoring operations.  For example,
while doing a rename operation, it over-approximates the set of
expressions that must be modified when a property is renamed in a safe
manner.
%FIXME explained enough?
% e.g: A set {\tt L} of object labels over-approximates a set {\tt O} of runtime objects if every
% object {\tt o $\in$ O} is represented by some {\tt l $\in$ L}.

To prove the concept, three refactoring operations were implemented,
namely rename, encapsulate property, and extract module.  By using
over-approximations it is possible to be sure when a refactoring
operation is valid.  However, this approach has the disadvantage of
not applying every possible refactoring operation, because the
refactoring operations for which the framework cannot guarantee
behavior preservation are prevented.  The wrongly prevented operations
accounts for 6.2\% of all rejections.
