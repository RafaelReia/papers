\section{Related Work}
\label{sec:Related-Work}

\subsection{Scheme}
A refactoring tool \cite{griswold1991program} for scheme, implemented in Lisp that
uses two forms of information, AST and PDG (program dependence Graph).
The AST represents the abstract syntactic structure of the program.
While the PDG explicitly represents the key relationship of dependence between
operations in the program. %control flow!
The graph vertices's represent program operations and the edges represent the flow of
data and control between operations.
However the PDG only has information of dependencies of the program. And if there
are two semantically unrelated statements they could be placed arbitrarily with
respect to each other.
%Whitout the constraints given by the AST two semantically unrelated statements
%could be placed arbitrarily with respect to each other. %rewrite!
Using the AST as the main representation of the program ensures that statements
are not arbitrarily reorder.
And the PDG is only used as a notation to prove that transformations preserve
the meaning and as quick way to retrieve needed dependence information.
Contours are used with the PDG providing scope information, non existent in the PDG,
to help reason about transformations in the PDG.
%Rewrite, talk about the scopes
With these structures it is possible to have a single formalism to reason effectively
about flow dependencies and scope structure.
\subsection{Haskell}
HaRe \cite{thompson2005refactoring} is a refactoring tool for Haskell that
integrates with Emacs and Vim.
%(abstract syntax tree)
The HaRe system uses an AST of the program to be
refactored in order to reason about the transformations to do.
The system has also a token stream in order to preserve the comments and the
program layout by keeping information about the source code location and the comments of all tokens.
It retrieves scope information from the AST, that allows to have refactoring operations that
require binding information of variables. %so a def-use-relation?
The system also allows the users to design their own refactoring operations
using the HaRe API.
\subsection{Python}
  \subsubsection{Rope}
  Rope is a Python refactoring tool written in Python, which works like a Python library.
  In order to make it easier to create refactoring operations, Rope assumes that a
  Python program only has assignments and functions calls. %(can this be a bad thing?)
  Thus, by limitating the complexity of the languages reduces the complexity of the
  refactoring tool.
  %The tool can easily get information about the assignments.
  %However, for function calls it is necessary to have other approaches in order to
  %obtain the necessary information.
  Rope uses a Static Object Analysis, which analyses the modules or scope to get
  information about functions.
  Rope only analyses the scopes when they change and it only analyses the modules
  when asked by the user, because this approach is time consuming.

  The other approach is the Dynamic Object Analysis that requires running
  the program in order to work.
  The Dynamic Object Analysis gathers type information and parameters passed to and returned from
  functions in order to get all the information needed.
  It stores the information collected by the analysis in a database.
  If Rope needs the information and there is nothing on the database the Static
  object inference starts trying to infer the object information.
  This approach makes the program run much slower, thus it is only active when
  the user decides.%TODO improve
  Rope uses an AST in order to store the syntax information about the programs.

  \subsubsection{Bicycle Repair Man}
  Bicycle Repair Man
  Bicycle Repair Man is a Refactoring Tool for Python written in Python.
   This refactoring tool can be added to IDEs and editors, such as Emacs, Vi, Eclipse,
    and Sublime Text.
  Bicycle Repair Man is an attempt to create the refactoring browser functionality for
   Python and has the following refactoring operations:
   extract method, extract variable, inline variable, move to module, and rename.

  The tool has an AST to represent the program and a database that has information
  about several program entities and dependency information.
  \subsubsection{Pycharm-edu} %%%ADD references
  Pycharm Educational Edition\footnote{https://www.jetbrains.com/pycharm-edu/},
   or Pycharm edu, is a IDE for Python created by JetBrains,
  the creator of IntelliJ.
  The IDE was specially designed for the educational purpose, for programmers
  with little or no previous coding experience.
  Pycharm EDU is a simpler version of Pycharm community which is the free
  python IDE created by JetBrains.
  Therefore t is very similar to their normal IDEs it has interesting features
  as code completion, version control tools integration.
  However it has a simpler interface when compared with
  Pycharm Community and other IDEs such as Eclipse or Visual Studio. %good for beginners, explain

  It has integrated a python tutorial and the big advantage is the possibility of
  the teachers creating tasks/tutorials for the students.
  However the Refactoring Tool did not received the same care as the IDE itself.
  The refactoring operations are exactly the as the Pycharm community IDE wich were made
  for more advanced users.
  Therefore it does not provide specific refactoring operations to beginners.
\subsection{Javascript}
There are few refactoring tools for JavaScript but there is a framework
\cite{feldthaus2011tool} for refactoring JavaScript programs. %FIXME I do not like it
In order to guarantee the correctness of the refactoring operation, the framework
uses preconditions, expressed as query analyses provided by pointer analysis. %TODO explain better!
Queries to the pointer analysis produces over-approximations of sets in a safe way to
have correct refactoring operations.
For example, while doing a rename operation, it over-approximates the set of expressions
that must be modified when a property is renamed in a safe manner.
%FIXME explained enough?
%a set L of object labels over-approximates a set O of runtime objects if every
%object o ∈ O is represented by some l ∈ L.
To prove the concept, three refactoring operations were implemented, namely rename,
encapsulate property, and extract module.
By using over-approximations it is possible to be sure when a refactoring
operation is valid.
However, this approach has the disadvantage of not applying every possible refactoring operation,
because the refactoring operations for which the framework cannot guarantee behavior
preservation are prevented.%FIXME rewrite if possible
It was expectable that this framework would prevent a big percentage of refactoring
operations, but after testing with 50 JavaScript programs,
the overall unjustified rejections were 6.2\% of all the rejections.
The rejections regarding imprecise preconditions represent 8.2\%.
Finally, the unjustified rejections due to imprecise pointer analysis were of
5.9\% for the rename and 7.0\% for the encapsulate property.
