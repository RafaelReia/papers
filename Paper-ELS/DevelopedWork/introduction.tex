\section{Introduction}
%language support for beginners
%refactoring is an advantage
%there is a lot of tools that provide refactoring operations to
%languages that are "pedagogical"
%Existing tools are not made for beginners
%Racket is a pedagogical language
%DrRacket is a pedagogical environment
%Sweet COMBO
%we take on a pedagogical environment that SUPPORTS pedagogical languages
%and we made an AWESOME refactoring tool for beginners.

%the need to know how to program
%the importance to know how to program+

%TODO!!
%TODO explain/define refactoring! Add Examples!
%%%%
%The need to need to know how to program and create simple programs is growing
%specially in areas non related with any computation field %computer engineering
%It is growing so abruptly that it is starting to appear job proposals requiring
%a double degree in architecture or design and in the computation field
%to work in a architecture studio. %because that knowledge does not come with the first course.
%Thus creating the need to have a better support to beginner programmers. %%really?

%value priority attention consideration influence interest
%that already are job offers that require a degree in any computation field.

Programing relevance as a skill is growing in areas non related with any computation field.  %specialy
This urge to know how to program as a complementary skill %craving
to the main degree demands better support tools for the novice programmers. %(formation)
We consider a novice/beginner programmer as a user who had one semester of programming class. %FIXME rewrite, weird step HELP
%Giving the users the tools and the means to create better software.


There are several languages known to be suited for the initial contact %better?
with programming, such as Scheme, Racket, Python and JavaScript which are used in introductory
courses around the world.
In addition, there are integrated development environments (IDEs) targeted for
 users with little or no previous contact with programming \cite{kolling2003bluej}. %system pedagogy 2.2 and 3.4
The pedagogically-aware %educational
 IDE provides the tools and the means to create better programs while simplifying
the complexity of a typical IDE \cite{pears2007survey}.%pedagogy 3.3 and 3.4  %examples? DrRacket and PyCharm Edu

One important module of an IDE is the embedded refactoring tool, %part/tool/element
which provides support to refactoring operations intended to improve the design
of an existing code base \cite{fowler1999refactoring} without changing the behavior of the program.
Languages used in introductory courses already have refactoring tools available,
however they were made for more advanced users and not for beginners.
%The absent of specific refactoring operations for beginners suggests that, in general,
%those refactoring tools do not have refactoring operations that
%would fit a beginner user. %FIXME weird AF
The lack of refactoring operations for beginner users makes those refactoring tools
unfit for a beginner.

A refactoring tool for beginner users needs to improve code made by them, %must help to improve / should help to improve / ?
with refactoring operations for the typical errors made, and simple enough to be
used by a beginner user.
Automatic detection would also help the users to know the refactoring operations
available and where they are applicable.

%to detect code that might/should
In contrast, the typical refactoring tools do not provide any support for the detection of code which might or should %FIXME is that correct?
be refactored.
On top of that, the IDEs in which most of those tools are embedded, such as Eclipse\cite{carlson2005eclipse},
 IntelliJ \cite{bock2011intellij}, NetBeans \cite{boudreau2002netbeans}, VisualStudio\cite{ford2011coding},
 Vim\cite{moolenaar2008vim}, Emacs\cite{stallman2007gnu}, are too complex for beginners
 to use, requiring the user to understand several complex menus do or to learn
 the special commands to properly use the IDE.
Therefore having a steep learning curve, which makes it difficult to beginners
explore the tool. %take advantage, fully use the tool, etc
%Whereas the Eclipse and Visual Studio are complex IDEs for beginners making it difficult
% for a beginner user to understand and explore the IDE.
For instance, Eclipse has around 300 menu bar options and Visual studio 280 witch is a massive amount
of options for the user to select.
On the other end, DrRacket has around 100. %(every option was accounted in the sub-menus %hand or end? even options that are not available).
Regardless the number of options available, the options available in Eclipse or
in Visual studio were in average more complex than the options available in DrRacket.

Provide a refactoring tool aimed (made) for beginners, students that have one semester %targeted, concerned %TODO improve
of programing classes, that helps to improve typical design errors made and in addition can
make suggestions showing the possible refactoring operations found.
%to help those users with possible refactoring operations.
Such refactoring tool brings a new set of options for the beginners to use
in order to safely improve their code and while they get used to a refactoring tool.
%FIXME


DrRacket, formely known as DrScheme is a pedagogical IDE \cite{drscheme} \cite{drscheme_pegadogy},
tailor made for the Racket programming language, which currently does not
have refactoring operations exept from the rename.
Such refactoring tool would be an extension to the DrRacket IDE that is already
used in several introductory courses, and known as a good language to learn how to program.

%give moto
%DrRacket Pedagogical:
%Catches typical syntactic mistakes of beginners and pinpoints the exact source location
%of run-time exceptions. %%more stuff??
%These languages already have refactoring tools available, however those refactoring tools
%are meant for advanced users only providing refactoring operations for more advanced
%users.
%to detect code that might/should
%And do not provide any support in the detection of code that might/should be refactored.
%On top of that, the IDE that those tools are inserted are too complex for beginners
%to use, such as Eclipse, VisualStudio, Vim, Emacs. %carefull here!

%Talk about Python, Javascript, Racket, lisp?
%There are already several refactoring tools made for languages used in introductory courses
%or that are simple enough to be considered tailored for beginners (or pedagogical ones)
%However there are none made for beginners, meaning that do not have refactoring operations
%that would fit a beginner user, namely for improving syntax made by beginners, and
%simple enough to be used by a beginner user.
