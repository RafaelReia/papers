\section{Introduction}

The support provided to beginner programmers is increasing and there are already
several languages known to be suited for the initial contact with programming, such as Scheme, Racket,
Python and JavaScript which are used in introductory courses around the world.
In addition, there are integrated development environments (IDEs) targeted for
 users with little or no previous contact with programming \cite{kolling2003bluej}. %system pedagogy 2.2 and 3.4
The pedagogically-aware %educational
 IDE provides the tools and the means to create better programs while simplifying
the complexity of a typical IDE \cite{pears2007survey}.%pedagogy 3.3 and 3.4

After learning the fundamentals of programming a beginner programmer should have the
capacity to transform the code systematically.
In order to help with that there are the refactoring tools,
which provides support to refactoring operations intended to improve the design
of an existing code base \cite{fowler1999refactoring} without changing the behavior of the program.
Languages used in introductory courses already have refactoring tools available,
however they were made for more advanced users and not for beginners.
The lack of refactoring operations for beginner programmers makes those refactoring tools
unfit for a beginner.

A refactoring tool for beginner programmers needs to improve code made by them, %must help to improve / should help to improve / ?
with refactoring operations for the typical errors made, and simple enough to be
used by a beginner programmers.
Automatic detection of possible refactorings would also help the users to know
which refactoring operations available and where they are applicable.

In contrast, the typical refactoring tools do not provide any support for the detection of code which might or should
be refactored.
On top of that, the IDEs in which most of those tools are embedded, such as Eclipse\cite{carlson2005eclipse},
 IntelliJ \cite{bock2011intellij}, NetBeans \cite{boudreau2002netbeans}, VisualStudio\cite{ford2011coding},
 Vim\cite{moolenaar2008vim}, and Emacs\cite{stallman2007gnu}, are too complex for beginners
 to use, requiring the user to understand several complex menus or to learn
 the special commands to properly use the IDE.
As a result this IDEs have a steep learning curve, which makes it difficult for beginners
to explore the tool. %take advantage, fully use the tool, etc
For instance, Eclipse has around 300 menu bar options and Visual studio 280, which is a massive amount
of options for the user to select.

In this paper we present a refactoring tool aimed for beginner programmers which typically
had one introductory course.
Our tool helps to detect the typical programming mistakes made by students and in
addition can make suggestions showing the possible refactoring operations found.
This refactoring tool brings a new set of options for the beginners to use
in order to safely improve their code.


We implemented our idea for a refactoring tool in DrRacket, formerly known as
DrScheme is a pedagogical IDE \cite{drscheme}\cite{drscheme_pegadogy}.
DrRacket was tailor made for the Racket programming language, which currently
has only one simple refactoring operation which allows renaming a variable.
Our implementation significantly extends the set of refactoring operations available
in Drracket.
