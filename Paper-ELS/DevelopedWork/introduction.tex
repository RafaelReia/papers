\section{Introduction}
In order to become a proficient programmer, one needs not only to
master the syntax and semantics of a programming language, but also
the style rules adopted in that language and, more important, the
logical rules that allow him to write simple and understandable
programs.  Given that beginner programmers have insufficient knowledge
about all these rules, it should not be surprising to verify that
their code reveals what more knowledgeable programmers call ``poor
style,'' ``bad smells,'' etc.  As time goes by, it is usually the case
that the beginner programmer learns those rules and starts producing
correct code written in an adequate style.  However, the learning
process might take a considerable amount of time and, as a result,
large amounts of poorly-written code might be produced before the end
of the process.  It is then important to speed up this learning
process by showing, from the early learning phases, how a
poorly-written fragment of code can be improved.

After learning how to write code in a good style, programmers become
critics of their own former code and, whenever they have to work with
it again, they are tempted to take advantage of the opportunity to
restructure it so that it conforms to the style rules and becomes
easier to understand.  However, in most cases, these modifications are
done without complete knowledge of the requirements and constraints
that were considered when the code was originally written and, as
result, there is a serious risk that the modifications might introduce
bugs.  It is thus important to help the programmer in this task so
that he can be confident that the code improvements he anticipate are
effectively applicable and will not change the meaning of the program.
This has been the main goal of \emph{code refactoring}.

Code refactoring is the process of changing a software system in such
a way that it does not alter the external behavior of the code yet
improves its internal structure \cite{fowler1999refactoring}.
Nowadays, any sophisticated IDE includes an assortment of refactoring
tools, e.g., for moving methods along a class hierarchy, to extract
interfaces from classes, and to transform anonymous classes into
nested classes.  It is important to note, however, that these IDEs
were designed for advanced programmers, and that the provided
refactorings require a level of code sophistication that is not
present in the programs written by beginners.  This makes the refactoring
tools inaccessible to beginners.
%For instance, Eclipse\cite{carlson2005eclipse} has around 300 menu bar options and VisualStudio\cite{ford2011coding} 280, which is a massive amount
%of options for the user to select.

In this paper, we present a tool that was designed to address the
previous problems.  In particular, our tool (1) is usable from a
pedagogical IDE designed for beginners
\cite{pears2007survey,kolling2003bluej}, (2) is capable of analyzing
the programmer's code and inform him of the presence of the typical
mistakes made by beginners, and finally (3) can apply refactoring
rules that restructure the program without changing its semantics.

To evaluate our proposal, we implemented a refactoring tool in
DrRacket, a pedagogical IDE \cite{drscheme,drscheme_pegadogy} used in
schools around the world to teach basic programming concepts and
techniques.  Currently, DrRacket has only one simple refactoring
operation which allows renaming a variable.  Our work significantly
extends the set of refactoring operations available in DrRacket and
promotes their use as part of the learning process.
