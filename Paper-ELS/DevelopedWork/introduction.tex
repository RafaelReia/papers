\section{Introduction}


%language support for begginers
%refactoring is an advantage
%there is a lot of tools that provide refactoring operations to
%languages that are "pedagogical"
%Existing tools are not made for beginners
%Racket is a pedagogical language
%DrRacket is a pedagogical enviromnent
%Sweet COMBO
%we take on a pedagogical environment that SUPPORTS pedagogical languages
%and we made an AWESOME refactoring tool for begginers.

The need to program is growing that other areas non related with software engineering
have the need to know how to program and create simple programs. It is growing
so abruptly that it is appearing job proposals that require double degree in
architecture or design and in the computation field.
This creates the need to have a better support to beginner programmers, by this
we meant a user that had one semester of programming class. Giving the users
the tools and the means to create better software.
There are several languages that said suited for learning and the initial contact
with programming, such as Racket, Python and JavaScript that are used in introductory
courses around the world.
And there are refactoring tools for such languages, however those refactoring tools
are meant for advanced users only providing refactoring operations for more advanced
users. And not giving any support to detect code that might/should be refactored.
On top of that, the IDE that those tools are inserted are too complex for beginners
to use, such as Eclipse, VisualStudio, Vim, Emacs. %carefull here!
Although there is DrRacket a pedagogical IDE, formely known as DrScheme, that was
build for the Racket Language, it does not have Refactoring Operations.


Extension to Racket language that is good for learning and used in several
programming introductory courses around the world.
It has a pedagogical IDE tailor made for the Racket programing language, namely
the DrRacket, former DrScheme.
Whereas the Eclipse and Visual Sutio are the evil IDEs with such level of
complexity that is really hard for beginner students to understand.
Eclipse has around 300 menu options and Visual studio 280 with is a massive amount
of options for the user to select, DrRacket for example has around 100. (every option was acounted in the submenus
even options that are not available). regardless the number of options available,
the options in Eclipse or in Visual studio were in average more complex than the
options available in DrRacket.
%DrRacket Pedagogical:
Catches typical syntactic mistakes of beginners and pipoints the exact source location
of run-time exceptions.

Provide a refactoring tool aimed (made) for beginers, students that have one semester
of programing classes, that help correct typical errors made and suggest them
with possible refactorings.
The refactoring tool brings a new set of options that the student could use to
get used to the refactoring tool and provide a way to improve their code without
inserting errors in the code.

%Talk about Python, Javascript, Racket, lisp?
There are already several refactoring tools mande for languages used in introductory courses
or that are simple enough to be considered tailored for beginners (or pedagogical ones)
However there are none made for beginners, meaning that do not have refactoring operations
that would fit a beginner user, namely for improving syntax made by begginners, and
simple enough to be used by a beginner user.

%give moto
%create story, but how?
%add examples of typical errors
