%Open source (confim that) Yes it is
%complexity not that bad, 100 options in the menu bar IT has part of the complexity in the proj options (more 20 options)
%Good stuff, embedded tutorial, bank of tutorials. possibility to teachers create tutorials to their students.
%Could have repl tough
%https://www.jetbrains.com/pycharm-edu/
%https://www.jetbrains.com/pycharm-edu/concepts/
%https://github.com/JetBrains/intellij-community/tree/master/python/src/com/jetbrains/python/refactoring
%PARSING
%https://github.com/JetBrains/intellij-community/tree/master/python/src/com/jetbrains/python/parsing
%it also has defuseutil relation.
%build in JAVA
\section{Pycharm}
Pycharm Educational Edition \footnote{https://www.jetbrains.com/pycharm-edu/},
 or Pycharm edu, is a IDE for Python created by JetBrains,
the creator of IntelliJ.
The IDE was specially designed for the educational purpose, for programmers
with little or no previous coding experience.
Pycharm EDU is a simpler version of Pycharm community which is the free
python IDE created by JetBrains.
Therefore t is very similar to their normal IDEs it has interesting features
as code completion, version control tools integration.
However it has a simpler interface when compared with
Pycharm Community and other IDEs such as Eclipse or Visual Studio. %good for beginners, explain

It has integrated a python tutorial and the big advantage is the possibility of
the teachers creating tasks/tutorials for the students.
However the Refactoring Tool did not received the same care as the IDE itself.
The refactoring operations are exactly the as the Pycharm community IDE wich were made
for more advanced users.
Therefore it does not provide specific refactoring operations to beginners.


%The goal of this refactoring tool is to integrate in a IDE aimed for
%educational purposes but provide a refactoring tool that is also aimed
%for beginners.

%altough it is a refactoring tool in a IDE aimed at beginners it does not
%have refactoring operations specific of beginners.
