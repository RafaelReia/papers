\section{FrameWork}
Python: [framework]
One of the goals of this thesis is to create refactoring for dynamic languages in general,
in other words, to create a Framework for refactoring tools.
//by using the same structure (program, framework, choose one better)
After creating several refactoring operations for Racket Python was chosen next.
POC (prove of concept)
It was created a prove of concept in Python consistent in several refactoring operations, such as:
[TODO] say which refactoring operations were implemented.
Python was chosen because there is already an implementation of Python for the
 DrRacket environment in which the refactoring tool for Racket was first developed.

how easy it is to add new refactoring operations and languages
The framework it is simply to use, it is only necessary to have a specification file
of each refactoring operation.
That file must have a function that receives two arguments,
one is the AST of the program and the other is the def-use-relation.
This information makes it possible to have several refactoring operations that help
the programmer.
what was "reused"
everything except the refactoring operations itself.
the advantages of that
This Framework makes it easier to implement refactoring operations for dynamic languages,
with only the catch that they have to be implemented for DrRacket. Helping minimizing
the problem of the difficulty and lacking of refactoring operations for dynamic languages.
(for at least every language implemented for DrRacket)

%[TODO Explain META-LANGUAGE]
The implementation for DrRacket is what does the trick, because by implementing
the language with Racket syntax we are basically using as a meta-language that
can represent several languages. Being the meta-language a language that are
composed only by syntax elements is a huge advantage to compute effortless new
refactoring operations for new languages when compared with the necessary effort to create
the refactoring operations directly for that language.
